\documentclass[11pt]{article}

\usepackage{amssymb}
\usepackage{amsmath}
\usepackage{setspace}
\usepackage{graphicx}
\usepackage{subfig}
\usepackage{color}
\usepackage[left=1.2in, right=1in, top=1in, bottom=1in]{geometry}
\usepackage{verbatim}
\usepackage{hyperref}

\begin{document}

\title{LORAAX Theory Guide}
\date{\today}
\maketitle

\tableofcontents

\section{Background}

LORAAX combines a 3D panel method, which assumes inviscid, irrotational flow,
with Xfoil, a 2D integral boundary layer code for flow over airfoils. Surface
velocity and pressure are computed via the 3D panel method, and Xfoil computes
the skin friction, displacement thickness, and parasitic drag at each spanwise
section. The equivalent 2D flow condition is specified in Xfoil via the local
sectional lift coefficient computed by integrating the surface pressure and
skin friction. The displacement thickness from Xfoil is then used to modify the
boundary condition for the inviscid solution via a transpiration velocity
applied at the 3D panel centroids. The transpiration velocity has the effect
of moving the normal-flow condition off the surface, where the real flow is
viscous and rotational, to the edge of the boundary layer. Due to the change in
boundary condition, the inviscid flow must be recomputed, which typically
affects the sectional lift coefficient. Therefore, LORAAX iterates between the
inviscid and boundary layer solutions until convergence is achieved. The
remainder of this document describes the 3D panel method and coupling with
Xfoil in detail.

\section{Inviscid Solution}\label{sec:inviscid}

\subsection{Governing Equation}

The linearized small-disturbance equation for irrotational, steady compressible
flow is
\begin{equation}
	(1 - M_\infty^2)\frac{\partial^2\Phi}{\partial x^2} +
	                \frac{\partial^2\Phi}{\partial y^2} +
	                \frac{\partial^2\Phi}{\partial z^2} = 0,
	\label{eq:small_disturbance}
\end{equation}

\noindent where $M_\infty$ is the Mach number in the freestream and $\Phi$ is
the velocity potential. $\Phi$ is related to the velocity vector by
\begin{equation}
	\mathbf{\nabla}\Phi = \mathbf{U}.
	\label{eq:velocity_potential}
\end{equation}

\noindent In other words, velocity is the gradient of the velocity potential. By
using the velocity potential formulation, the original system of partial
differential equations (the Navier-Stokes equations governing fluid flow) have
been reduced to a single partial differential equation. Such simplifications
are, of course, accompanied by several limitations, including:

\begin{itemize}
	\item Irrotational flow: boundary layers and shocks cannot be captured by
		this governing equation. In LORAAX, the boundary layer is computed
		separately and coupled with the inviscid solution via the boundary
		condition; see Section \ref{sec:viscous_coupling}.
	\item Small disturbances: the induced velocities must be much smaller in
		magnitude than the freestream velocity.
		This assumption sometimes implies thin surfaces at small angles of
		attack. However, the underlying \textit{incompressible} solution
		procedure is not subject to this assumption, and in practice acceptable
		results can be obtained even for thick surfaces near stall, from
		incompressible through low subsonic conditions.
\end{itemize}

Significant computational advantages can be obtained by transforming Eq.
\ref{eq:small_disturbance} into the Laplace equation. The Prandtl-Glauert
transformation does just that. To perform this transformation, we define a new
set of coordinates, $x_i$, $y_i$, and $z_i$, as follows:
\begin{align}
	\label{eq:prandtl_glauert_transform}
	x_i & = \frac{x}{\sqrt{1 - M_\infty^2}} \\ \nonumber
	y_i & = y \\ \nonumber
	z_i & = z. \nonumber
\end{align}

\noindent $x_i$, $y_i$, and $z_i$ are also known as \textit{incompressible
coordinates}, because the procedure transforms Eq. \ref{eq:small_disturbance}
into the Laplace equation, which governs irrotational, incompressible flows
(and is not limited to small disturbances). The Laplace equation is
\begin{equation}
	\frac{\partial^2\Phi_i}{\partial x_i^2} +
	\frac{\partial^2\Phi_i}{\partial y_i^2} +
	\frac{\partial^2\Phi_i}{\partial z_i^2} = 0.
	\label{eq:laplace}
\end{equation}

Equation \ref{eq:laplace} represents an incompressible flow equivalent to the
compressible flow of Eq. \ref{eq:small_disturbance}. This equation can be solved
using the well-known technique of superposition of elementary solutions,
including source and doublet panels and vortex rings, and results in a linear
system of equations to determine the strengths of these elementary solutions.
This process will be described in more detail in
Section \ref{sec:discretization}.

Notice that the
transformation described in Eq. \ref{eq:prandtl_glauert_transform} produces
an equivalent incompressible planform with the same span but greater chord and
sweep than the real compressible planform. Due to these effects, in three
dimensions it is not possible to simply solve Eq. \ref{eq:laplace} for the
original geometry and then scale the pressure coefficient by
$\frac{1}{\sqrt{1 - M_\infty^2}}$ to correct for compressibility as
can be done in two dimensions. The geometric transformation must actually be
applied and the problem solved for the equivalent incompressible geometry. In
LORAAX, the incompressible geometry is always computed as a preprocessing step,
and Eq. \ref{eq:laplace} is solved in incompressible coordinates.

The velocity potential for the either the incompressible or compressible problem
can be separated into a freestream and induced part as follows:
\begin{align}
	\label{eq:phi_sep}
	\Phi_i(x_i,y_i,z_i) & = \Phi_i'(x_i,y_i,z_i)
	                      + U_\infty(x_i\cos\alpha + z_i\sin\alpha) \\ \nonumber
	\Phi(x,y,z) & = \Phi'(x,y,z)
	              + U_\infty(x\cos\alpha + z\sin\alpha) \nonumber
\end{align}

\noindent where $\alpha$ is the angle of attack. $\Phi_i'$ is related to the
compressible induced potential by
\begin{equation}
	\Phi_i'(x_i,y_i,z_i) = \Phi'(x,y,z).
	\label{eq:phiprime_incompressible}
\end{equation}

\noindent Therefore, it is sufficient to solve the incompressible problem and
then apply Eqs. \ref{eq:phi_sep} and \ref{eq:phiprime_incompressible} to obtain
the solution in the compressible coordinate space. The velocity for the
compressible problem can be obtained in a similar manner.
Substituting Eqs. \ref{eq:phi_sep} and \ref{eq:phiprime_incompressible} into
Eq. \ref{eq:velocity_potential}, the compressible velocity can be
obtained anywhere in the flowfield by differentiating the
incompressible velocity potential as follows:
\begin{align}
	\label{eq:velocity_compressible}
	\mathbf{U} & = \left(\frac{\partial\Phi_i'}{\partial x} +
	                    U_\infty\cos\alpha\right)\hat{i} +
	              \frac{\partial\Phi_i'}{\partial y}\hat{j} +
	              \left(\frac{\partial\Phi_i'}{\partial z} +
	                    U_\infty\sin\alpha\right)\hat{k} \\ \nonumber
	           & = \left(\frac{\partial\Phi_i'}{\partial x_i}
	                    \frac{1}{\sqrt{1 - M_\infty^2}} +
	                    U_\infty\cos\alpha\right)\hat{i} + 
	              \frac{\partial\Phi_i'}{\partial y_i}\hat{j} +
	              \left(\frac{\partial\Phi_i'}{\partial z_i} +
	                    U_\infty\sin\alpha\right)\hat{k}. \nonumber
\end{align}

Similarly, the compressible pressure coefficient is related to the
incompressible pressure coefficient by
\begin{equation}
	c_p = \frac{c_{p_i}}{\sqrt{1 - M_\infty^2}},
	\label{eq:cp_compressible}
\end{equation}

\noindent but again, this correction assumes that Eq. \ref{eq:laplace} was
solved in incompressible coordinates to obtain $c_{p_i}$. Other local properties
can be computed via the isentropic flow relations.

\subsection{Discretization}\label{sec:discretization}

The biggest advantage of the potential flow governing equation
(Eq. \ref{eq:laplace}) is that the solution can be constructed by linearly
combining elementary solutions on the surfaces and wakes only. The linearity of
the governing equation makes it very easy to construct a linear system of
equations from the discretized geometry, and the fact that only surfaces and
wakes need to be modeled makes it much less computationally expensive than
methods that require the fluid volume to be discretized. Furthermore, the
partial derivatives in Eq. \ref{eq:laplace} do not need to be approximated
numerically as part of the discretization process, because the elementary
solutions are exact solutions of the governing equation. (However, there is
still discretization error in the final solution of any non-trivial problem,
because the elementary solutions do not represent the actual geometry nor the
actual circulation distribution exactly.)

LORAAX uses a combination of source and doublet panels to represent wing
surfaces and wakes. Both quadrilateral and triangular panels are employed.
The underlying solutions assume that each of these elements is planar, though
the quadrilateral elements on the wing surface may be slightly twisted without
appreciable loss of accuracy. Because wake panels may undergo significant
twist during the unsteady rollup procedure, the deforming panels in the wake
are triangular, which are guaranteed to remain planar even when their defining
vertices move. Surface panels include both source (to model wing and boundary
layer thickness) and doublet (to provide circulation and lift) elements, while
wake panels include only doublets. There is also a separate set of
non-deforming wake panels containing only source elements, which are responsible
for modeling the boundary layer dissipation in the wake.

\subsubsection{Wing Modeling}

Geometric discretization in LORAAX is fairly straightforward in concept. The
user defines a number of sections, each of which are defined by their location
in space, twist, and local dihedral angle. Sections are assumed to be aligned
with the x-direction prior to applying twist and dihedral rotations. The user
also defines one or more airfoils along the span of the wing. These airfoils
are discretized into a number of vertices on the top and bottom surfaces (the
number and spacing being defined by the user) and then interpolated to the
section spanwise section locations. Vertices of adjacent sections are connected
to form quadrilateral panels on the top and bottom surfaces of the wing. The
wingtips are assumed to be flat and are discretized by connecting the top and
bottom surface tip vertices and dividing into two along the mean camber line.
Triangular panels are used at the leading and trailing edge tip panels where the
geometry forms a point, and quadrilateral panels are used in between. Blunt
trailing edges are not modeled; if the user-supplied airfoils have blunt
trailing edges, LORAAX will remove the trailing edge thickness by smoothly
modifying the airfoil shape beginning at the 90\% chord location. However, the
original blunt trailing edge is retained when running Xfoil to obtain the
parasitic drag and boundary layer properties.

\subsubsection{Wake Modeling}

The wake is modeled as a thin sheet of doublet panels initially extending along
the direction parallel to the freestream. The user can elect to allow the wake
to deform as part of the solution, in which case a rolling-up distance is
specified along with the number of deforming panels that should be placed along
this distance. As was mentioned previously, triangular panels are used for the
deforming portion of the wake to ensure that each panel remains planar. Behind
the deforming portion of the wake, a single quadrilateral panel is placed along
the freestream direction and extending to infinity. (Since only finite
lengths are allowed for doublet panels, ``infinity'' is actually 750 times the
maximum span of any wing in the aircraft, which produces results that are
indistinguishable from truly infinite panel. Induced velocity calculations are
performed by treating wake panel edges as vortex elements, and in this case the
trailing elements are assumed to actually be infinite.) If the user chooses to
allow the wake to deform, a time-stepping approach is employed to convect the
wake vertices, which is described in detail in Section \ref{sec:rollup}. The
geometry of the wake panels is then recomputed after convection occurs. The
doublet strengths for the wake panels are a function of the doublet strength
distribution on the wing surface, which is described in Section
\ref{sec:system}.

The viscous wake is treated as a separate set of panels including only source
elements. The vertex locations for these panels comes from Xfoil, and they do
not convect. The viscous wake influence on the surface is considered during the
solution procedure, but it is ignored during wake rollup calculations. The
source strengths for the viscous wake are a function of the boundary layer
displacement thickness and edge velocity, which is described in Section
\ref{sec:viscous_coupling}.

\subsubsection{Elementary Solutions}

LORAAX uses a first-order panel representation, which means that each source and
doublet element has constant strength across its surface assumed-planar surface.
The induced potential due to a doublet panel at a point $(x_i, y_i, z_i)$ can
therefore be represented by
\begin{equation}
	\Phi_d(x_i,y_i,z_i) = c_d(x_i,y_i,z_i)\mu_d,
	\label{eq:doublet}
\end{equation}

\noindent where $c_d$ is the potential due to a unit-strength doublet panel,
also known as an \textit{influence
coefficient}, and $\mu_d$ is the doublet strength of the panel. Similarly, the
induced potential due to a source panel is
\begin{equation}
	\Phi_s(x_i,y_i,z_i) = c_s(x_i,y_i,z_i)\sigma_s,
	\label{eq:source}
\end{equation}

\noindent where $\sigma_s$ is the source strength. Induced velocities have a
similar form, except that there are three influence coefficients per panel
(representing
the $x_i$-, $y_i$-, and $z_i$-components of the velocity due to a unit strength
element). When using the vortex representation to compute the induced velocity
of a doublet element, the Biot-Savart law is employed, treating each edge of
the panel as a vortex leg with circulation strength equal to the doublet
strength of the element. The equations for the influence coefficients are
very long; LORAAX uses the equations from Katz and Plotkin
\textcolor{red}{(citation)}, so the reader can refer to that text or read the
source code (note that a few of the equations in Ref.
\textcolor{red}{(citation)} needed to be corrected). The Biot-Savart law and its
implementation for finite and semi-infinite vortex legs are also available from
the same reference.

As the point $(x_i, y_i, z_i)$ moves far away from the panel, the influence of
the panel becomes essentially the same as a point source or doublet element
located at the panel's centroid.
Since the influence coefficients for point elements are much cheaper to
compute, panels are treated as point elements whenever the point is sufficiently
far away. Through testing, a distance of 8 characteristic panel lengths was
found to be sufficiently accurate while significantly reducing the computational
time. Here, the characteristic length is the length of the diagonal for
quadrilateral panels and the longest side length for triangular panels.

\subsection{Boundary Conditions}

\subsection{Construction of the Linear System}\label{sec:system}

The Neumann-type boundary condition specifies 0 normal flow through a panel:

\begin{equation}
\mathbf{V}\cdot{\mathbf{\hat{n}}} = 0,
\label{eq:bc1}
\end{equation}

\noindent where $\mathbf{\hat{n}}$ is the outward-facing normal vector at the centroid of
the panel. The velocity is the sum of contributions from all source and doublet panels on
wing surfaces, doublet panels in the wake, plus the freestream influence, as follows:

\begin{equation}
\mathbf{V}(x,y,z) = \mathbf{V}_\infty
                  + \sum_{i=1}^{\text{pan}}\left(\mathbf{a}_i(x,y,z)\mu_i
                  +                              \mathbf{b}_i(x,y,z)\sigma_i\right)
                  + \sum_{i=\text{pan}+1}^{\text{wakepan}}\mathbf{a}_i(x,y,z)\mu_i,
\label{eq:vel1}
\end{equation}

\noindent where $\mathbf{a}_i$ and $\mathbf{b}_i$ are the influence
coefficients due to the
doublet panel $i$ and source panel $i$, respectively, and $\mu$ and $\sigma$ are
the strengths of these elements, assumed constant on a panel.

Equation \ref{eq:vel1} gives the boundary condition in the Neumann form. For the purpose
of evaluating it numerically, it is sometimes preferred to represent the boundary
condition in Dirichlet form. In this case, it is recognized that Eq. \ref{eq:vel1} is
equivalent to the velocity potential, $\Phi$, being constant inside the body. $\Phi$ is
related to the velocity as follows:

\begin{equation}
\Phi = \nabla\mathbf{V}.
\label{eq:phi1}
\end{equation}

For the source-doublet panel formulation,

\begin{equation}
\Phi(x,y,z) = \Phi_\infty
            + \sum_{i=1}^{\text{pan}}\left[c_i(x,y,z)\mu_i + d_i(x,y,z)\sigma_i\right]
            + \sum_{i=\text{pan}+1}^{\text{wakepan}+\text{pan}}c_i(x,y,z)\mu_i,
\label{eq:phi2}
\end{equation}

\noindent where $c_i$ and $d_i$ are the influence coefficients for the velocity potential
of doublet and source panels $i$, respectively acting on the point $(x,y,z)$. Note that in
this formulation, the influence coefficients are scalars instead of vectors as in Eq.
\ref{eq:vel1}. The boundary
condition states that $\Phi$ is constant inside the body, so

\begin{equation}
   \Phi_\infty
 + \sum_{i=1}^{\text{pan}}\left[c_i(x,y,z)\mu_i + d_i(x,y,z)\sigma_i\right]
 + \sum_{i=\text{pan}+1}^{\text{wakepan}+\text{pan}}c_i(x,y,z)\mu_i = \text{const},
\label{eq:phi3}
\end{equation}

\noindent where it is assumed that the point $(x,y,z)$ lies inside the body. Different
values for the constant are possible, but a popular choice is $\text{const} =
\Phi_\infty$, which results in the boundary condition

\begin{equation}
\sum_{i=1}^{\text{pan}}\left[c_i(x,y,z)\mu_i + d_i(x,y,z)\sigma_i\right]
\sum_{i=\text{pan}+1}^{\text{wakepan}+\text{pan}}c_i(x,y,z)\mu_i = 0.
\label{eq:phi4}
\end{equation}

This choice of the internal potential requires the source strengths to be

\begin{equation}
\sigma_i = -\mathbf{V}_\infty\cdot\mathbf{\hat{n}}_i,
\label{eq:sigma1}
\end{equation}

\noindent where $\mathbf{\hat{n}}_i$ is the outward-pointing normal direction at
the centroid of panel $i$ \textcolor{red}{(cite Katz \& Plotkin)}.

Finally, the Kutta condition must be enforced, which states that no jump in $\Phi$ is
allowed at the trailing edge. This is done by setting the doublet strength in the wake
panels along the trailing edge as

\begin{equation}
\mu_{\text{te}} = \mu_{\text{te,top}} - \mu_{\text{te,bot}},
\label{eq:wake_strength}
\end{equation}

\noindent where \emph{te} represents the portion of the wing's trailing edge
spanned by a given column of surface and trailing wake panels.

\section{System of Equations}

Applying Eq. \ref{eq:phi4} at every surface panel, and substituting in Eqs.
\ref{eq:sigma1} and \ref{eq:wake_strength}, Eq. \ref{eq:system1} is obtained.
Here, the identifier $(x,y,z)$ has been removed, as it is implicit that this equation is
to be applied at the centroid of a panel, just inside the body.

\begin{equation}
  \sum_{i=1}^{\text{pan}}c_i\mu_i 
+ \sum_{k=1}^{N_\text{te}}e_k\left(\mu_{k,\text{top}} - \mu_{k,\text{bot}}\right)
= \sum_{i=1}^{\text{pan}}d_i\mathbf{V}_\infty\cdot\mathbf{\hat{n}}_i
\label{eq:system1}
\end{equation}

\noindent Note that the unknowns are the doublet strengths $\mu$ only, while the
known quantities have been moved to the right hand side. The influence
coefficients $e_k$ represent the potential due to the strip of wake doublets
behind the trailing edge spanwise location $k$ with unit strength
Equation \ref{eq:system1} can be
applied to each of the surface panels, with only the influence coefficients
$c$, $e$, and $d$ changing. The result is a system of
$N_{\text{pan}}$ equations with $N_{\text{pan}}$ unknown doublet strengths. It
can be represented in matrix form as follows:

\begin{equation}
\mathbf{\overline{AIC}}\boldsymbol{\mu} = \mathbf{RHS},
\label{eq:system2}
\end{equation}

\noindent where $\mathbf{\overline{AIC}}$ is the matrix of aerodynamic influence
coefficients,
$\boldsymbol{\mu}$ is the vector of unknown doublet strengths, and $\mathbf{RHS}$ is the
vector of known quantities on the right hand side of the equation. $\mathbf{\overline{AIC}}$
is a
dense square matrix, so a direct method like LU decomposition is most appropriate to solve
it.

\section{Viscous - Inviscid Coupling}\label{sec:viscous_coupling}

Don't forget to mention the viscous wake.

\section{Computation of Forces and Moments}

\section{Wake Rollup}\label{sec:rollup}

\end{document}
